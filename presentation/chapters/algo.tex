\section{The different algorithm implementations}

% ------------------------------------------------------------------------------
\begin{frame}
\frametitle{Agenda}
\tableofcontents[currentsection]
\end{frame}
% ------------------------------------------------------------------------------


% ------------------------------------------------------------------------------
\begin{frame}
\frametitle{Flink algorithm 1}
\begin{figure}
	
	\includegraphics[scale=0.2]{dataartisians.png}
	\caption{dataArtisians logo, \cite{artisians}}
\end{figure}
\begin{itemize}
\item An exercise from dataArtisians
\item Uses the standard Gelly implementation
\item \# input nodes = \# output nodes
\end{itemize}
\end{frame}
% ------------------------------------------------------------------------------

% ------------------------------------------------------------------------------
\begin{frame}
\frametitle{Flink algorithm 2}
\begin{figure}
	
	\includegraphics[scale=0.2]{dataartisians.png}

\end{figure}

\begin{itemize}
\item A case study implementation from dataArtisians
\item A custom implementation
\item \# input nodes = \# output nodes
\end{itemize}
\end{frame}
% ------------------------------------------------------------------------------

% ------------------------------------------------------------------------------
\begin{frame}
\frametitle{Flink algorithm 3}
\begin{figure}

	\includegraphics[scale=0.05]{squirrel.png}

\end{figure}

\begin{itemize}
\item An example from the Apache Flink repository
\item A custom implementation
\item \# input nodes != \# output nodes $\rightarrow$ filters
\end{itemize}
\end{frame}
% ------------------------------------------------------------------------------


% ------------------------------------------------------------------------------
\begin{frame}
\frametitle{Turi pagerank algorithm}
\begin{figure}
	\centering
	\includegraphics[scale=0.5]{turi.png}
	\caption{Turi logo, \cite{turi}}
\end{figure}
\begin{itemize}
\item Used the standard implementation
\item Builds a graph out of the edges dataset
\end{itemize}


\end{frame}
% ------------------------------------------------------------------------------